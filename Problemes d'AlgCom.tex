\documentclass[11pt, a4paper]{article}

\usepackage{graphicx} % Required for inserting images
% \usepackage[utf8]{inputenc} % Que caracteres se van a ver
\usepackage[document]{ragged2e} % Poner los párrafos
\usepackage{amsmath,amsthm,amssymb,latexsym}
\usepackage{enumitem}
% \usepackage{xcolor}
\usepackage[x11names]{xcolor}
\usepackage{tikz}
\usepackage{tcolorbox}
\usepackage{tkz-euclide}
\usepackage{pgfplots}
% \usepackage{cancel}
\usepackage{hyperref}
\usepackage{mathrsfs,mathtools}
\usepackage{microtype}
\usepackage{fancyhdr}
\usepackage{parskip}
\usepackage[catalan]{babel}

\renewcommand\refname{}
\usepackage[margin=1in]{geometry}



\pgfplotsset{compat=1.18}
\usetikzlibrary{patterns}

\setlength{\headheight}{15.21004pt}

\DeclareMathOperator{\Ima}{Im}

%% Macros
\providecommand{\ol}{\overline}
\providecommand{\ul}{\underline}
\providecommand{\wt}{\widetilde}
\providecommand{\wh}{\widehat}
\providecommand{\eps}{\varepsilon}
\providecommand{\half}{\frac{1}{2}}
\providecommand{\inv}{^{-1}}
\newcommand{\dang}{\measuredangle} %% Directed angle
\providecommand{\CC}{\mathbb C}
\providecommand{\FF}{\mathbb F}
\providecommand{\NN}{\mathbb Z_{\geq 1}}
\providecommand{\QQ}{\mathbb Q}
\providecommand{\RR}{\mathbb R}
\providecommand{\PP}{\mathbb P}
\providecommand{\ZZ}{\mathbb Z}
\providecommand{\dd}{\mathrm d}
\newcommand{\KK}{\mathbb{K}}
\newcommand{\TT}{\mathcal{T}}
\providecommand{\ts}{\textsuperscript}
\providecommand{\dg}{^\circ}
\providecommand{\ii}{\item}
\DeclareMathOperator*{\lcm}{lcm}
\DeclareMathOperator*{\argmin}{arg min}
\DeclareMathOperator*{\argmax}{arg max}
\DeclareMathOperator*{\rad}{rad}
\providecommand{\bigO}{\mathcal O}




% theorem environments
% starred versions are not numbered, unstarred versions have a number
\theoremstyle{definition}
    \newtheorem{theorem}{Teorema}
    \newtheorem{lemma}[theorem]{Lema}
    \newtheorem{claim}[theorem]{Claim}
    \newtheorem{corollary}[theorem]{Corol·lari}
    \newtheorem{definition}[theorem]{Definició}
    \newtheorem{proposition}[theorem]{Proposició}
    \newtheorem{example}[theorem]{Exemple}
    
    \newtheorem*{theorem*}{Teorema}
    \newtheorem*{lemma*}{Lema}
    \newtheorem*{claim*}{Claim}
    \newtheorem*{corollary*}{Corol·lari}
    \newtheorem*{definition*}{Definició}
    \newtheorem*{proposition*}{Proposició}
    \newtheorem*{example*}{Exemple}
    \newtheorem*{proof*}{Demostració}
    
\theoremstyle{remark}
    \newtheorem{remark}[theorem]{Remarca}
    \newtheorem*{remark*}{Remarca}


\providecommand{\bigO}{\mathcal O}
\usepackage{hyperref}
\renewcommand{\thesubsection}{}
\renewcommand{\thesection}{}

\title{Introducció a l'àlgebra commutativa}
\author{Bernat Esteve Sagarra}

\begin{document}
\pagestyle{fancy}
% 
\lhead{Bernat Esteve}
\rhead{Problemes d'Àlgebra commutativa}
\begin{tcolorbox}[title=\section{Exercici 1}]
    Sigui $A$ un anell, $I$ un ideal. Recordeu que es defineix el \textit{radical} de $I$, $\rad(I)=\eta(I)$, com el conjunt d'elements $x\in A$ tals que existeixen $n\in\NN$ tal que $x^n\in I$. Proveu que:
    \begin{enumerate}
        \item $\rad(I)$ és la intersecció de tots els ideals primers de $A$ que contenen $I$.
        \item Si $I$ és un ideal primer, $\rad(I^n)=I$ per a tot $n>0$.
        \item $\rad(I)$ és un ideal maximal si, i només si, $I$ està contingut en un únic ideal primer.
    \end{enumerate}
\end{tcolorbox}
\subsection{Exercici 1.1}
Per veure el primer apartat, notem que $\rad(I)$ és un ideal. Per veure-ho notem que $\rad(I)/I$ són els elements nilpotents de $A/I$; i per tant és un ideal (per la bijecció entre un conjunt i l'altre).\par
\textbf{Primera inclusió:}\\
Per fer la primera inclusió, hem de veure que $\rad$ és un ideal primer. Sigui $ab\in\rad(I)$, aleshores sabem que $(ab)^n\in I$ per alguna $n$. Per tant, si fem quocient per $I$, tenim que $ab$ és nilpotent, i per tant, ha de ser al $\rad(I)$.\par
\textbf{Segona inclusió:}\\
Sigui $J$ un ideal primer que conté $I$, aleshores tenim $x^n\in I \implies x\in I$; és a dir que el $\rad\subseteq J$, que és el que volíem veure.
\subsection{Exercici 1.2}
Notem que $I^2\subseteq I$, per tant, $\rad(I^2)\subseteq\rad(I)$, per tant, en tenim prou en veure-ho per $\rad(I)$. Però sabem que $A/I$ on $I$ és un ideal primer és un domini d'integritat, i per tant, no hi ha elements nilpotents, i per tant el radical és $I$, que és el que volíem veure.
\subsection{Exercici 1.3}
\textbf{Primera implicació:}\\
Per veure la primera implicació, notem que si $I$ només està contingut en un ideal primer $J$, aquest ha de ser maximal. I com que el radical ha de ser un ideal primer que contingui $I$, aquest haurà de ser $J$.\par
\textbf{Segona implicació:}\\
Per veure la segona implicació, notem que si l'ideal $\rad$ és maximal, aleshores també és primer; i com que $\rad$ ha de viure dins de tots els ideals primers que contenen $I$, aquest ha de ser únic.

\end{document}
% ---
% \newpage
% \begin{tcolorbox}[title=\section{Exercici XXX}]
% \end{tcolorbox}
% \subsection{Exercici XXX}
