
\usepackage{graphicx} % Required for inserting images
% \usepackage[utf8]{inputenc} % Que caracteres se van a ver
\usepackage[document]{ragged2e} % Poner los párrafos
\usepackage{amsmath,amsthm,amssymb,latexsym}
\usepackage{enumitem}
% \usepackage{xcolor}
\usepackage[x11names]{xcolor}
\usepackage{tikz}
\usepackage{tcolorbox}
\usepackage{tkz-euclide}
\usepackage{pgfplots}
% \usepackage{cancel}
\usepackage{hyperref}
\usepackage{mathrsfs,mathtools}
\usepackage{microtype}
\usepackage{fancyhdr}
\usepackage{parskip}
\usepackage[catalan]{babel}

\renewcommand\refname{}
\usepackage[margin=1in]{geometry}



\pgfplotsset{compat=1.18}
\usetikzlibrary{patterns}

\setlength{\headheight}{15.21004pt}

\DeclareMathOperator{\Ima}{Im}

%% Macros
\providecommand{\ol}{\overline}
\providecommand{\ul}{\underline}
\providecommand{\wt}{\widetilde}
\providecommand{\wh}{\widehat}
\providecommand{\eps}{\varepsilon}
\providecommand{\half}{\frac{1}{2}}
\providecommand{\inv}{^{-1}}
\newcommand{\dang}{\measuredangle} %% Directed angle
\providecommand{\CC}{\mathbb C}
\providecommand{\FF}{\mathbb F}
\providecommand{\NN}{\mathbb Z_{\geq 1}}
\providecommand{\QQ}{\mathbb Q}
\providecommand{\RR}{\mathbb R}
\providecommand{\PP}{\mathbb P}
\providecommand{\ZZ}{\mathbb Z}
\providecommand{\dd}{\mathrm d}
\newcommand{\KK}{\mathbb{K}}
\newcommand{\TT}{\mathcal{T}}
\providecommand{\ts}{\textsuperscript}
\providecommand{\dg}{^\circ}
\providecommand{\ii}{\item}
\DeclareMathOperator*{\lcm}{lcm}
\DeclareMathOperator*{\argmin}{arg min}
\DeclareMathOperator*{\argmax}{arg max}
\DeclareMathOperator*{\rad}{rad}
\providecommand{\bigO}{\mathcal O}




% theorem environments
% starred versions are not numbered, unstarred versions have a number
\theoremstyle{definition}
    \newtheorem{theorem}{Teorema}
    \newtheorem{lemma}[theorem]{Lema}
    \newtheorem{claim}[theorem]{Claim}
    \newtheorem{corollary}[theorem]{Corol·lari}
    \newtheorem{definition}[theorem]{Definició}
    \newtheorem{proposition}[theorem]{Proposició}
    \newtheorem{example}[theorem]{Exemple}
    
    \newtheorem*{theorem*}{Teorema}
    \newtheorem*{lemma*}{Lema}
    \newtheorem*{claim*}{Claim}
    \newtheorem*{corollary*}{Corol·lari}
    \newtheorem*{definition*}{Definició}
    \newtheorem*{proposition*}{Proposició}
    \newtheorem*{example*}{Exemple}
    \newtheorem*{proof*}{Demostració}
    
\theoremstyle{remark}
    \newtheorem{remark}[theorem]{Remarca}
    \newtheorem*{remark*}{Remarca}


\providecommand{\bigO}{\mathcal O}